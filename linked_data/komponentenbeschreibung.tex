\section{Dokumentengeschichte}
\begin{table}[h]
 \begin{tabular}{|l|l|p{4cm}|}
 \hline
 Zeitraum & PL/Autor(en) & Änderungen \\
 \hline
 Wintersemester 2017/2018 & Georg Grauberger & 
Aufgabe der Komponente \newline
Architektur \newline
genutzte Komponenten
 \\
 \hline
 Wintersemester 1980/81 & IHR NAME & 
 \\
 \hline
 \end{tabular}
 \caption{Dokumentengeschichte}
 \end{table}

\section{Aufgabe der Komponente}
Die Aufgabe der Linked Data Schnittstelle ist es die OHDM Datenbank ueber
eine REST Schnittstelle im WWW verfuegbar zu machen.\newline
Die Daten sollen dann aufeinander verweisen so das man durch diese navigieren
kann. Die Objekte die durch die Schnittstelle freigegeben werden sind Geoobjekte
im geoJson Format.
\section{Architektur}

\subsection{Überlick}
Grafik der Teile der Komponente (wichtig: Benennung aller Schnittstellen). 
Anwendung der Komponente nennen (Use Case).

Übliche Interaktionen durch Interaktionsdiagramme.

Eine Spezifische Architektur fuer diese Komponente gibt es nicht, da wir die Richtlinien
des Spring-Boot Frameworks verwenden.
Als grobe zusammenfassung kann man jedoch sagen das das spring-boot das MVC Muster verwendet.
Wir implementieren also mit unserem Code die Model Schicht indem wir ein Mapping
fuer die Tabellen erstellen.
Die View und der Controller werden von dem maven Plugin 'spring-boot-starter' erstellt.
Das passiert automatisch da 'spring-boot' im Sinne von CoC agiert.

\subsection{Schnittstellendefinitionen}
Beschreibung der angebotenen Schnittstellen. Benennung der Funktionen
mit Vor- und Nachbedingungen. Beschreibuung des Protocol-Bindings.

(Beginnen in Prototyp-Phase. Konkretisieren in der Alphaphase)

\subsection{genutzte Komponenten}

\begin{table}[h]
\begin{tabular}{|l|p{4cm}|}
   \hline 
   Komponente & Beschreibung \\
   \hline
   \href{https://docs.spring.io/spring-data/jpa/docs/current/reference/html/}{spring-boot-data-jpa} &
    Ist verantwortlich fuer die Datenschicht, gibt Zugriff auf die Java Persistence API
   \\ \hline
   \href{https://docs.spring.io/spring-data/rest/docs/current/reference/html/}{spring-boot-starter-data-rest} &
   Beinhaltet viele automatismen um schnell eine REST Schnittstelle zu schaffen.
   \\ \hline
   \href{https://docs.spring.io/spring-data/rest/docs/current/reference/html/#_the_hal_browser}{spring-data-rest-hal-browser} &
   Beinhaltet einen REST Api browser mit dem man die Schnittstelle testen kann.
   \\ \hline
   \href{https://docs.spring.io/spring-boot/docs/current/reference/htmlsingle/#using-boot-starter}{spring-boot-starter-tomcat} &
   Beinhaltet einen Tomcat Server so das man zum Lokalen Testen und Starten keine einzelstehende Instanz
   vom Tomcat braucht.
   \\ \hline
   \href{https://docs.spring.io/spring-boot/docs/current/reference/html/using-boot-devtools.html}{spring-boot-devtools} &
   Beinhaltet Konfiguration fuer Quickreload und Quickdeployment in der Entwicklungsumgebung.
   \\ \hline
   postgresql &
   Beinhaltet den JDBC Treiber fuer PostGres.
   \\ \hline
\end{tabular}
\end{table}
\section{Nutzung}
\subsection{Code}
Wo findet man den Code. Struktur des Codes. (In Prototyphase ausfüllen,
kann dort sehr kurz sein. Ab Alpha-Phase konkret beschreiben.)

\subsection{Deployment / Runtime}
Beschreibung wie die Komponenten aus dem Quellcode erzeugt werden kann,
wie sie installiert wird und wie man sie startet.

\section{Qualitätssicherung}
(Ausfüllen ab Alpha-Phase).

Wie erfolgt die Sicherung der Qualität? Keine Romane, sondern ehrlich notieren,
was man tut. Wenn man nichts tut, dann steht hier: Wir sichern die Qualität der
Komponente nicht.

Issue-Tracking: wie erfolgt das, interne Fehlermeldungen (ab Alpha), 
externe Fehlermeldungen ab Beta.

\subsection{Test}
Wie wird die Komponente getestet.

\section{Vorschläge / Ausblick}
Was ist aufgefallen, was sollte man ändern? Löschen Sie auch gern die Kommentare
der Vorgänger, aber nur, wenn es wirklich nicht mehr relevant ist.

