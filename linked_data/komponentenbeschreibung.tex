\graphicspath{ {img/} }

\section{Dokumentengeschichte}
\begin{table}[h]
 \begin{tabular}{|l|l|p{4cm}|}
 \hline
 Zeitraum & PL/Autor(en) & Änderungen \\
 \hline
 Wintersemester 2017/2018 & Georg Grauberger & 
Aufgabe der Komponente \newline
Architektur \newline
genutzte Komponenten
 \\
 \hline
 Wintersemester 1980/81 & IHR NAME & 
 \\
 \hline
 \end{tabular}
 \caption{Dokumentengeschichte}
 \end{table}

\section{Aufgabe der Komponente}
Die Aufgabe der Linked Data Schnittstelle ist es die OHDM Datenbank ueber
eine REST Schnittstelle im WWW verfuegbar zu machen.\newline
Die Daten sollen dann aufeinander verweisen so das man durch diese navigieren
kann. Die Objekte die durch die Schnittstelle freigegeben werden sind Geoobjekte
im geoJson Format.
\section{Architektur}

\subsection{Überlick}
Grafik der Teile der Komponente (wichtig: Benennung aller Schnittstellen). 
Anwendung der Komponente nennen (Use Case).

Übliche Interaktionen durch Interaktionsdiagramme.

Eine Spezifische Architektur fuer diese Komponente gibt es nicht, da wir die Richtlinien
des Spring-Boot Frameworks verwenden.
Als grobe zusammenfassung kann man jedoch sagen das das spring-boot das MVC Muster verwendet.
Wir implementieren also mit unserem Code die Model Schicht indem wir ein Mapping
fuer die Tabellen erstellen.
Die View und der Controller werden von dem maven Plugin 'spring-boot-starter' erstellt.
Das passiert automatisch da 'spring-boot' im Sinne von CoC agiert.

\subsection{Schnittstellendefinitionen}
Beschreibung der angebotenen Schnittstellen. Benennung der Funktionen
mit Vor- und Nachbedingungen. Beschreibuung des Protocol-Bindings.

(Beginnen in Prototyp-Phase. Konkretisieren in der Alphaphase)

\subsection{genutzte Komponenten}

\begin{table}[h]
\begin{tabular}{|l|p{4cm}|}
   \hline 
   Komponente & Beschreibung \\
   \hline
   \href{https://docs.spring.io/spring-data/jpa/docs/current/reference/html/}{spring-boot-data-jpa} &
    Ist verantwortlich fuer die Datenschicht, gibt Zugriff auf die Java Persistence API
   \\ \hline
   \href{https://docs.spring.io/spring-data/rest/docs/current/reference/html/}{spring-boot-starter-data-rest} &
   Beinhaltet viele automatismen um schnell eine REST Schnittstelle zu schaffen.
   \\ \hline
   \href{https://docs.spring.io/spring-data/rest/docs/current/reference/html/#_the_hal_browser}{spring-data-rest-hal-browser} &
   Beinhaltet einen REST Api browser mit dem man die Schnittstelle testen kann.
   \\ \hline
   \href{https://docs.spring.io/spring-boot/docs/current/reference/htmlsingle/#using-boot-starter}{spring-boot-starter-tomcat} &
   Beinhaltet einen Tomcat Server so das man zum Lokalen Testen und Starten keine einzelstehende Instanz
   vom Tomcat braucht.
   \\ \hline
   \href{https://docs.spring.io/spring-boot/docs/current/reference/html/using-boot-devtools.html}{spring-boot-devtools} &
   Beinhaltet Konfiguration fuer Quickreload und Quickdeployment in der Entwicklungsumgebung.
   \\ \hline
   postgresql &
   Beinhaltet den JDBC Treiber fuer PostGres.
   \\ \hline
\end{tabular}
\end{table}
\section{Nutzung}
\subsection{Code}
Den Code der Schnittstelle kann man auf GitHub über den Link finden: \\
\url{https://github.com/OpenHistoricalDataMap/LOD} \\
\\
Für die Realisierung der Schnittstelle wurde MVC-Framework Spring-Boot 
verwendet, dadurch hat der Code folgende Struktur: \\
\\
\includegraphics[scale=0.7]{image} \\
\\
\\
Alle notwendige Einstellungen für die Verbindung mit dem GeoServer sind 
in der Datei \textbf{application.properties} zu finden.
Modelle und deren Repositorien liegen entsprechend in Verzeichnissen 
\textbf{model} und \textbf{repository}. \\
\\
\\
\\
Struktur von \textbf{application.properties}:

\lstset{frame=single,basicstyle=\ttfamily\footnotesize,breaklines=true}

\begin{lstlisting}
# Datenquelle
spring.datasource.url = jdbc:[Driver]://[Host]:[Port]/[Datenbank]
spring.datasource.username = [Benutzername]
spring.datasource.password = [Passwort]

# Datenbank beim Startup initializieren
spring.jpa.hibernate.ddl-auto = [none|validate|update|create|create-drop]

spring.jpa.properties.hibernate.dialect = [Dialekt]
spring.jpa.properties.hibernate.default_schema = [DefaultSchema]
\end{lstlisting}
\vspace{5mm}
\noindent
Beispielhafte Einstellungen in \textbf{application.properties}:

\begin{lstlisting}
spring.datasource.url = jdbc:postgresql://htw.de:8000/test
spring.datasource.username = mustermann
spring.datasource.password = admin1234

spring.jpa.hibernate.ddl-auto = none
spring.jpa.properties.hibernate.dialect = org.hibernate.dialect.PostgreSQL9Dialect
spring.jpa.properties.hibernate.default_schema = ohdm
\end{lstlisting}

\subsection{Deployment / Runtime}
Das LOD-Server kann lokal durch die Kommandozeile mit folgenden Befehle aus 
dem Root-Verzeichnis des Projektes gestartet werden:\\
\\
Windows (mit cmd.exe):
\begin{lstlisting}
mvnw.cmd spring-boot:run
\end{lstlisting}
Linux und macOS:
\begin{lstlisting}
mvn spring-boot:run
\end{lstlisting}
Das lokale Server wird dann über die URL \textbf{127.0.0.1:8080} 
(oder localhost:8080) verfügbar.\\
\\
Um die Anwendung auf ein Webserver zu installieren, muss man zuerst eine WAR-Datei 
(Web Application Resource) erzeugen. Dafur stehen folgene Befehle zur Verfügung: \\
\\
Windows (mit cmd.exe):
\begin{lstlisting}
mvnw.cmd package
\end{lstlisting}
\vspace{10mm}
\noindent
Linux und macOS:
\begin{lstlisting}
mvn package
\end{lstlisting}
\vspace{5mm}
\noindent
Nach dem Ausführen des Befehls wird eine WAR-Datei im Ordner \textbf{target} 
des Root-Verzeichnises erstellt. Diese Datei kann dann auf ein Tomcat-Webserver
hochgeladen und ausgeführt werden.


\section{Qualitätssicherung}
(Ausfüllen ab Alpha-Phase).

Wie erfolgt die Sicherung der Qualität? Keine Romane, sondern ehrlich notieren,
was man tut. Wenn man nichts tut, dann steht hier: Wir sichern die Qualität der
Komponente nicht.

Issue-Tracking: wie erfolgt das, interne Fehlermeldungen (ab Alpha), 
externe Fehlermeldungen ab Beta.

\subsection{Test}
Wie wird die Komponente getestet.

\section{Vorschläge / Ausblick}
Was ist aufgefallen, was sollte man ändern? Löschen Sie auch gern die Kommentare
der Vorgänger, aber nur, wenn es wirklich nicht mehr relevant ist.

