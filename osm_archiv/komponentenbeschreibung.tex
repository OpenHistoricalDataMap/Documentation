\section{Dokumentengeschichte}
\begin{table}[h]
	\begin{tabular}{|l|l|p{4cm}|}
		\hline
		Zeitraum & PL/Autor(en) & Änderungen \\
		\hline
		Wintersemester 17/18 & Schwotzer, Thomas & XML-Datei von OSM parsen und diese Daten in die Intermediate-DB speichern\\
		\hline
		Wintersemester 17/18 & Wendt, Jan-Derk & Optimierung und Anpassung des XML-Default-Handlers an die neue Postgres-COPY-Anbindung\\
		\hline
		Wintersemester 17/18 & Sauer, Florian & Implementation einer Datenbankanbindung zu Intermediate-DB mittels Postgres-COPY \\
		\hline
	\end{tabular}
	\caption{Dokumentengeschichte}
\end{table}

\section{Aufgabe der Komponente}
Eine, wenn nicht die, wesentliche Quelle für OHDM ist Open Street 
Map (OSM)\footnote{osm.org}.
Eine, wenn nicht die, wesentliche Quelle für OHDM ist Open Street Map (OSM)\footnote{osm.org}.

In einem {\it initialen Upload} wurde OHDM im Sommer 2017 mit den Daten des Planet.osm Files vom Januar 2017 gefüllt.

Diese Komponente realisiert ein optional erneutes initiales Einlesen der XML-Dateien und das jährliche Update der Intermediate-Datenbank basierend auf den XML-Dateien von OSM.

Der Einlese- und Update-Prozess wird im Detail auf den nächsten Seiten beschrieben.

\section{Architektur}
Die Komponente teilt sich in zwei Subkomponenten:

\begin{description}
\item[OSM2Intermediate] 
parsed das OSM File und füllt die
Intermediate Database.

\item[Intermediate2OHDM] füllt oder erneuert die OHDM Datenbank 
mit Daten aus OSM.

\end{description}

Diese Komponenten bietet keine Schnittstelle nach außen an.

Diese Komponente nutzt keine weiteren Komponenten des Systems.

\section{OSM-to-Intermediate}
Diese Teilkomponenten parsed die OSM Files und füllt die Intermediate Database.
Die Struktur der Intermediate ist einfach. Sie enthält fünf Tabellen.

Die Tabelle {\tt nodes}, {\tt ways} und {\tt relations} werden direkt aus
den Einträgen im OSM-File gefüllt. Jede Tabelle enthält OSM-Nutzer und -ID.
Die Nodes enthalten die Koordinaten. Ways und Relations enthalten die IDs
der Nodes bzw. Ways, die den Way bzw. die Relation beschreiben.

Die IDs werden in diesen Tabelle als String gehalten. Dadurch wird die
Reihenfolge der IDs gespeichert.

Es gibt zwei weitere Tabelle: {\tt waynodes} repräsentiert die 1-n Beziehung
zwischen ways und nodes. Die {\tt relationsmember} speichert die 1-n-Beziehung 
zwischen Relation und ihren Membern (nodes bzw. ways.)

\subsection{SQL\_OSMImporter}
Der {\tt SQL\_OSMImporter} implementiert {\tt DefaultHandler} und arbeitet wie folgt:

\subsubsection{Begin / Ende Dokument}
Der Parser erkennt den Beginn und das Ende des XML Dokuments. Diese Events werden jeweils
einmal am Anfang und am Ende des Parse-Prozesses geworfen. Die Methoden {\tt startDocument()}
und {\tt endDocument()} werden dabei aufgerufen. Bei Beginn wird eine Statusmeldung erzeugt.

Am Ende werden die {\tt SQLQueues} geschlossen - siehe dazu \ref{SQLStatementQueue}.

\subsubsection{Start / End Element}
Der Parser ruft die Methode {\tt startElement()} auf, wenn er den Beginn eines XML
Tags erkennt. Die Methode {\tt endElement()} wird gerufen, wenn das Ende eines
XML Elements entdeckt wird. XMl-Elemente können geschachtelt sein und sind es im
OSM-XML-File auch. Einem Aufruf eines {\tt startElement()} können weitere Aufrufe
der gleichen Methode folgen, weshalb der Zustand relevant ist, in dem der Aufruf erfolgt.

Die beiden Methoden werden durch den Importer implementiert. Es gibt sechs verschiedene
Tags im OSM File. Die Tags {\tt node, way, relation} enthalten Beschreibungen von Punkten,
Wegen oder Relationen. Die Tags {\tt tag, nd, member} treten nur innerhalb der Tags
auf.

Die ersten drei Tags dürfen nur als direkte Kindknoten der XML-Root
auftauchen. Es wird deshalb geprüft, ob der Parser aktuell {\it außerhalb - OUTSIDE}
war, d.h. auf der Ebene der Root. Es ist ein Fehler, wenn das nicht der Fall ist.
Der Fehler wird aber ignoriert, was nicht sauber programmiert ist (!).

Im Erfolgsfall wird für {\tt node, way, relation} die Methode {\tt newElement} aufgerufen.
Im Fall von {\tt tag, nd, member} wird jeweils {\tt addAttributes, addND, und addMember} aufgerufen.

Das Tag {\tt tag} enthält weitere Informationen zu dem Element - das sind Attribute,
die später in die Intermediate DB eingetragen werden. Das Tag {\tt nd} gibt es nur innerhalb
von {\tt way} Tags. Es folgt die ID eines Nodes, das Teil des Weges ist. Das Tag {\tt member}
gibt es nur innerhalb einer {\tt relation}. Es folgen Beschreibungen (vor allem IDs)
der Member einer Relation. Das können Nodes und Ways sein.

\subsubsection{newElement}
Mit jedem Aufruf von {\tt newElement} wird ein {\tt INSERT} Kommando erzeugt.
Dieses Kommando wird nicht direkt an die Datenbank geschickt, sondern in einer 
{\tt SQLStatementQueue} gepuffert, siehe \ref{SQLStatementQueue}. Das dient lediglich
der Performance.

In dieser Implementierung werden parallel mehrere {\tt SQLStatementQueues} gefüllt.
Die Insert-Queue enthält {\tt INSERT}-Statements, die die Tabellen {\tt node, ways, relations}
der Intermediate DB füllen. Die Member-Queue sammelt Statements, die in die {\tt waynodes, relationmember}
gespeichert werden.

Der Code mag anfangs etwas verwirrend sein. Es hilft, zu verfolgen, wie die verschiedenen
Queues nacheinander gefüllt werden. Es ist auch zu beachten, dass die Statements erst mit
dem Aufruf von {\tt endElement} geschlossen werden. 

Es gilt auch zu beachten, dass zwischen den Start und dem Ende eines Elements auch
die anderen drei Methoden {\tt addAttributes, addND, und addMember} aufgerufen werden
können, die die {\tt INSERT}-Statement im weitere Parameter ergänzen.

\subsection{COPY\_OSMImporter}
Der {\tt COPY\_OSMImporter} implementiert {\tt DefaultHandler} und ist zuständig für die Funktion und das Vorgehen des XML-Parsers.

Die Funktionsweise ist ähnlich dem {\tt SQL\_OSMImporter}. Im Folgenden werden wahrscheinliche einige Aspekte wie oben erläutert. Jedoch sind auch einige Sachen neu und angepasst an die neue Situation.

Die XML-Datei von OSM ist folgendermaßen aufgebaut:

Es gibt 3 Arten von Hauptelementen und zwar Node, Way, Relation. Innerhalb dieser Hauptelemente können je nach Typ noch folgende innere Subelemente vorkommen: Tags, Waymember(nd), Relationmember(member).

Tags können in jedem Hauptelement vorkommen, wobei Waymember(nd) nur innerhalb von Ways und Relationmember(member) nur innerhalb von Relations vorkommen können.

Jedes Hauptelement hat mehr oder weniger gleiche Attribute, wie z.B. ID, User, User-ID. Es gibt auch elementspezifische Attribute wie z.B. die Koordinaten bei den Nodes.

\subsubsection{grobe Funktionsweise}
Der Handler des XML-Parsers fängt die öffnenden und schließenden Tags der 6 Elementarten ab und kann so auf diese Daten zugreifen und mit eigener Struktur formatieren, damit diese daraufhin vom {\tt CopyConnector} in die Intermediate-DB gespeichert werden können.

Die Klasse {\tt COPY\_OSMImporter} ist der Handler des XML-Parsers und erbt von DefaultHandler. Die grobe Arbeitsweise ist im folgenden kurz erläutert.

\subsubsection{Begin / Ende Dokument}
Der Parser erkennt den Beginn und das Ende des XML Dokuments. Diese Events werden jeweils
einmal am Anfang und am Ende des Parse-Prozesses geworfen. Die Methoden {\tt startDocument()}
und {\tt endDocument()} werden dabei aufgerufen. Bei Beginn wird eine Statusmeldung erzeugt.

Am Ende wird die Anzahl von eingelesenen Hauptelementen ausgegeben.

\subsubsection{Start / End Element}
Der Parser ruft die Methode {\tt startElement()} auf, wenn er den Beginn eines XML
Tags erkennt. Die Methode {\tt endElement()} wird gerufen, wenn das Ende eines
XML Elements entdeckt wird. XML-Elemente können geschachtelt sein und sind es im
OSM-XML-File auch. Einem Aufruf eines {\tt startElement()} können weitere Aufrufe
der gleichen Methode folgen, weshalb der Zustand relevant ist, in dem der Aufruf erfolgt.

Die beiden Methoden werden durch den XML-Handler implementiert. Es gibt sechs verschiedene
Tags im OSM File. Die Tags {\tt node, way, relation} enthalten Beschreibungen von Punkten,
Wegen oder Relationen. Die Tags {\tt tag, nd, member} treten nur innerhalb der Tags
auf.

Die ersten drei Tags dürfen nur als direkte Kindknoten der XML-Root
auftauchen. Es wird deshalb geprüft, ob der Parser aktuell {\it außerhalb - OUTSIDE}
war, d.h. auf der Ebene der Root. Es ist ein Fehler, wenn das nicht der Fall ist.

Im Erfolgsfall wird für {\tt node, way, relation} die Methode {\tt startMainElement()} aufgerufen. Im Fall von {\tt tag, nd, 
member} wird die Methode {\tt startInnerElement()} aufgerufen. In beiden Methoden folgen noch weiter Schritte und Unterscheidungen, welche u.A. im Quellcode selbst und im Javadoc dokumentiert sind.

Das Tag {\tt tag} enthält weitere Informationen zu dem Element - das sind Attribute, 
die später serialisiert mit dem jeweiligen Element in die Intermediate DB eingetragen werden. 
Das Tag {\tt nd} gibt es nur innerhalb von {\tt way}-Tags und gibt die ID einer Node an, welche Teil des jeweiligen Ways ist. 
Das Tag {\tt member} gibt es nur innerhalb einer {\tt relation} und gibt die ID einer Node, eines Ways oder einer Relation an, welche Teil der jeweiligen Relation ist.

\subsubsection{startMainElement}
Mit jedem Aufruf von {\tt startMainElement()} werden zwei formatierte Strings im csv-Format erstellt, welche nach und nach mit Inhalten gefüllt werden. Ein String enthält die Inhalte des jeweiligen aktiven Hauptelements Node, Way oder Relation. Der andere String enthält die ID's der jeweiligen zugehörigen Member (Waymember, Relationmember).

Es ist zu beachten, dass die Strings erst mit dem Aufruf von {\tt endMainElement()} fertig gefüllt werden.

Es gilt auch zu beachten, dass zwischen dem Start und dem Ende eines Hauptelements Methoden wie {\tt startInnerElement()} aufgerufen werden, welche die formatierten Strings mit weiteren Inhalten füllen.

\subsubsection{endMainElement}
Sobald das Ende eines Hauptelements erreicht ist, sind alle Inhalte für ein Hauptelement im formatierten String hinterlegt oder in Teilvariablen verfügbar um in den fertigen String übergeben zu werden.

In dieser Methode und in der Methode {\tt startInnerElement()} wird der jeweils fertig formatierte String in 5 verschiedenen Situation (Node, Way, Relation, Waymember, Relationmember) an den jeweiligen {\tt CopyConnector} übergeben, welcher eine Methode ausführt, wo der formatierte String per COPY-Befehl in die SQL-DB geschrieben wird.

\subsubsection{Schnittstellendefinitionen}
Da der Handler nur für diese konkrete Situation spezialisiert ist und keine allgemeinnützigen Funktionen bereitstellt, bietet diese Teilkomponente keine Schnittstellen nach außen an.

\subsubsection{genutzte Komponenten}
Der XML-Handler nutzt neben der Klasse {\tt CopyConnector} zwei weitere Klassen, welche diverse Hilfsmethoden zur Organisation und Einordnung der XML-Elemente zur Verfügung stellen.

Die beiden Klassen {\tt OSMClassificationCopyImport} und {\tt UtilCopyImport} sind mit vorhanden und müssen nicht extra bezogen werden. Eine grobe Funktionsweise und Übersicht ist im Quellcode selbst und im Javadoc zur Verfügung gestellt.

Darüber hinaus nutzt diese Teilkomponente keine weiteren Komponenten des Systems.

\section{Utilities}
\subsection{SQLStatementQueue}
\label{SQLStatementQueue}
Objekte von {\tt SQLStatementQueue} sind ein Puffer zwischen dem Parser/Handler und
der Datenbank. Objekte der {\tt SQLStatementQueue} werden mit einem Parameterfile erzeugt.
In dem File stehen die wesentlichen Informationen, um eine JDBC-Connection zu einer Datenbank
zu erzeugen.

Danach arbeiten sie ähnlich einem {\tt StringBuilder}. Es können schrittweise mit {\tt append}
String hinzugefügt werden. Die Objekte prüfen nicht, ob eine gültige SQL-Syntax entsteht.
Die Objekte senden die Statement an die Datenbank, wenn ein definierbarer Schwellwert erreicht
ist oder wenn explizit die Methode {\tt force} (in Varianten) aufgerufen wird.

Eine Variante sind die FileSQLQueues. Diese erzeugen Files, in denen die Statements gespeichert
werden. Die Managed-Queues sorgen außerdem dafür, dass diese Files nach einem gewissen Füllstand
geschlossen werden und mittels psql ausgeführt werden. 

Die Implementierung dieser Klassen ist sehr stabil. {\bf Der Nutzung hat sich bewährt und ist in
dieser Komponente Pflicht!}

\subsection{CopyConnector}
Der {\tt CopyConnector} ist ein Objekt, welches die Anbindung zur Intermediate Datenbank mithilfe des, in Postgres existierenden, 'COPY' Befehl herstellt.
Der Vorteil des Postgres-'COPY' Befehls ist es, dass er sich fadür eignet, große mengen an Daten in die Datenbank zu importieren. Im Gegensatz zu einem einfügen mittels INSERT, muss hier nicht immer die Syntaktische Korrektheit des SQL-Befehls überprüft werden. Eine abschließende Überprüfung der Korrektheit der eingefügten Daten findet dennoch statt. Dies ist jedoch um ein vielfaches effizienter. 

Der {\tt CopyConnector} benötigt zur Initialisierung einen Parameterfile, welcher neben Informationen zur Datenbank (Host, User, Passwort, Schema) auch Informationen zu den Tabellen selbst und den benötigten Spaltennamen enthält. Da die Tabellen der Intermediate Datenbank automatisch generierte IDs besitzen, müssen manche Tabellenspalten ignoriert werden bzw die benötigten explizit ausgewählt werden.
Um Daten Reihenweise zur Datenbank zu schreiben, müssen diese in einem CSV formatierten Format zusammengefügt werden. Hier ist zu beachten, dass Sonderzeichen (Trennzeichen, Leerzeichen, Zeilenschaltungen, etc) korrekt escaped oder passend verpackt werden.

\subsection{Parameter}
{\tt Parameter} ist ein Objekt, das zum Parsen und speichern von Konfigurationsdateien genutzt wird. Um es zu initialisieren, muss eine configurationsdatei zum einlesen angegeben werden.
Diese muss so strukturiert sein, dass pro zeile immer ein Key-Value paar existiert.
Der Key wird jeweils mit einem Kolon (Doppelpunkt / ':') von einer Value abgetrennt.

\begin{table}[h]
 \begin{tabular}{|l|p{10cm}|}
 \hline
 Key Name & Beschreibung \\
 \hline
 servername & Die Hostadresse zum benutzenden Postgres Server (String)\\
 \hline
 portnumber & Die Portnummer zum benutzenden Postgres Server (Integer)\\
 \hline
 username & Der Username des zu Benutzenden Useraccounts des Postgres Servers (String)\\
 \hline
 pwd & Das Passwort für den angegebenen Useraccount (String)\\
 \hline
 dbname & Der Name der zu benutzenden Datenbank (String)\\
 \hline
 schema & Das zu benutzende Schema in der ausgewählten Datenbank (String)\\
 \hline
 connectionType & Der Typ der Anbindungsart, Verfügbar sind: 'copy', um die neue CopyConnector\-Schnittstelle zu benutzen; 'insert', um die alte SQL Insert Schnittstelle zu benutzen\\
 \hline
 delimiter & Das zu benutzende Trennzeichen, genutzt für die generierung von CSV Strings im CopyConnector\\
 \hline
 nodesColumnNames & Die zu benutzenden Spaltennamen der 'nodes'\-Tabelle. Die Spaltennamen werden mit | von einander abgetrennt.
 Standartmäßig sollte der wert 'osm\_id|tstamp|classcode|otherclasscodes|serializedtags|longitude|latitude|has\_name|valid' eingetragen werden.\\
 \hline
 relationmemberColumnNames & Die zu benutzenden Spaltennamen der 'relationmember'\-Tabelle. Die Spaltennamen werden mit | von einander abgetrennt.
 Standartmäßig sollte der wert 'relation\_id|node\_id|way\_id|member\_rel\_id|role' eingetragen werden.\\
 \hline
 relationsColumnNames & Die zu benutzenden Spaltennamen der 'relations'\-Tabelle. Die Spaltennamen werden mit | von einander abgetrennt.
 Standartmäßig sollte der wert 'osm\_id|tstamp|classcode|otherclasscodes|serializedtags|member\_ids|has\_name|valid' eingetragen werden.\\
 \hline
 waynodesColumnNames & Die zu benutzenden Spaltennamen der 'waynodes'\-Tabelle. Die Spaltennamen werden mit | von einander abgetrennt.
 Standartmäßig sollte der wert 'way\_id|node\_id' eingetragen werden.\\
 \hline
 waysColumnNames & Die zu benutzenden Spaltennamen der 'ways'\-Tabelle. Die Spaltennamen werden mit | von einander abgetrennt.
 Standartmäßig sollte der wert 'osm\_id|tstamp|classcode|otherclasscodes|serializedtags|node\_ids|has\_name|valid' eingetragen werden.\\
 \hline
 \end{tabular}
 \caption{Parameter Key Beschreibungen}
 \end{table}
 
%% todo fix table


\section{Intermediate-to-OHDM}
Der Quellcode dieser Teilkomponenten liegt im package {\tt osm2inter}.

Das Package enthält nur wenige Klassen. {\tt OSMImport} enthält die {\tt main()}
Funktion. Dort wird ein {\tt SAXParser} erzeugt. Der Parser benötigt ein
Objekt, das die Klasse {\tt DefaultHandler} implementiert.

Der Parser parsed darauf das OSM-File. Sobald ein neues XML-Element gefunden wurde,
wird eine entsprechende Methode auf dem DefaultHandler aufgerufen. 



\section{Nutzung}
Der Code befindet sich im Repository {\tt OSMUpdateInsert}\footnote{https://github.com/OpenHistoricalDataMap/OSMImportUpdate}

\subsection{Code}
Der {\tt COPY\_OSMImporter} ist in Java geschrieben und ist im package {\tt osm2inter} im Respository von Github zu finden. Die zusätzlichen Hilfsklassen und allen nötigen Abhängigkeiten(Dependencies), zusätzlichen Libraries und weiteren Tools sind im package {\tt util} und {\tt osm} im Respository von Github.

\subsection{Deployment / Runtime}
Der {\tt COPY\_OSMImporter} kann nicht ohne weiteren Code ausgeführt werden. Er ist sozusagen eine Art Library, welche für einen XML-Parser verwendet werden kann. In unserem Fall startet die Klasse {\tt OSM\_Import} den XML-Parser und benutzt den {\tt COPY\_OSMImporter}.

\section{Qualitätssicherung}

\subsection{Fehlercatching / Sicherheit}
Der {\tt COPY\_OSMImporter} überprüft jegliche Art von Fehlern, ungewollten Situationen, falschen oder nicht vorhandenen Werten / Inhalten. Nur in den jeweiligen Best-Cases wird fortgefahren um falsche oder fehlende XML-Inhalte zu vermeiden. Sobald kein Best-Case vorhanden ist wird ein Standartoutput mit Log, Status und Zeilennummer der XML-Datei getätigt.

\subsection{Test}
Beim {\tt COPY\_OSMImporter} wurde während der Implementierungsphase mehrere Male im Adhoc-Stil eine XML-Datei absichtlich mit Fehlern gefüllt um zu gucken was passiert, wenn z.B. das Adminlevel keine gerechte Integerzahl beinhaltet, oder wenn die Schachtelung der Tags nicht ordnungsgemäß vorhanden ist.

In diesen Fällen wird alles ordnungsgemäß behandelt, was bedeutet das z.B. bei einer nicht gerechten Integerzahl das AdminLevel wieder auf 0 gesetzt wird. Bei allen vorkommenden XML-Fehlern wird versucht, dass der XML-Parser weiterläuft und die Fehler lediglich loggt. Sobald jedoch die Schachtelung / Struktur der XML-Datei fehlerhaft ist werden die jeweiligen Inhalte übersprungen, damit keine falschen Inhalte in die DB gelangen können. Dies wird natürlich auch geloggt jedoch nicht weiter behandelt, was dazu führt, dass Inhalte der XML-Datei nicht verarbeitet werden.

\section{Vorschläge / Ausblick}
Die Logik des {\tt COPY\_OSMImporter} an sich ist schon relativ weit ausgereift. Es werden alle möglichen Tag-Arten und Inhalte erkannt und verarbeitet.

Es sollte jedoch noch etwas am Fehlerhandling geändert / verbessert werden. Vor allem sollte eine andere / weitere Bewältigungsart bei Strukturfehlern in der XML-Datei entwickelt werden, damit keine Inhalte der XML-Datei übersprungen werden und somit erstmal verloren gehen. Auch wenn die XML-Dateien von OSM zu hoher Wahrscheinlichkeit keine Strukturfehler haben muss man trotzdem mit allen Situationen rechnen.

Zudem sollte der XML-Handler in so wenig wie möglichen wenn nicht sogar in garkeinen Situationen abbrechen / beenden, was jedoch bei den geworfenen SQL-Exceptions vom {\tt CopyConnector} der Fall ist.

