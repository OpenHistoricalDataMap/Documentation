\section{Dokumentengeschichte}
\begin{table}[h]
 \begin{tabular}{|l|l|p{4cm}|}
 \hline
 Zeitraum & PL/Autor(en) & Änderungen \\
 \hline
 Wintersemester 2017/18 & Thomas Schwotzer & 
 Initialer Text
 
  \\
 \hline
 Wintersemester 1980/81 & IHR NAME & 
text \newline 
text \newline 
 
  \\
 \hline
 
 
 \end{tabular}
 \caption{Dokumentengeschichte}
 \end{table}

\section{Ziel des Systems}
Open Historical Data Map (OHDM) versteht sich als offene und freie Plattform zur Speicherung, 
Abfrage und Visualisierung orts- und zeitgebundener Daten.

Primär sind damit {\bf Karten} gemeint. OHDM erlaubt die Speicherung von Vektorgeometrien
und generiert basierend darauf Karten, die über einen Web-Map-Service (WMS) angeboten werden,
siehe Kapitel \ref{wms_wfs}.

Das zentrales Konzept aber von OHDM ist das {\bf historische Geoobjekt}. Ein Geoobjekt
ist ein menschliches Artefakt, das zu einem oder mehrere Zeitpunkten einem Raum auf der
Erdoberfläche zugeordnet werden kann. 

Zum Management der Geometrien und der Geoobjekte entstehen eigene Editoren für OHDM.

Geoobjekten lassen sich weitere Daten zuordnen, vor allem Bilder, aber auch Sensordaten. 

OHDM Nutzer sollen historische Daten in das System laden können und diese dann per WMS 
anschauen können. OHDM wird eine Linked Data Schnittstelle anbieten und wir arbeiten an einem
SPARQL Endpoint.

\subsection{Historische Karten}
OHDM-Geoobjekte sind damit alle Gebäude, Straßen etc. die von Menschen geschaffen wurde.
Aber auch Verwaltungsgrenzen jeder Art sind Geoobjekte, ebenso wie Ereignisse, die sich 
Orten zuordnen lassen (Schlachten, Versammlungen, Revolutionen etc. pp.)

In jedem Fall können damit Geoobjekten Geometrie zugewiesen werden. Diese Zuordnung ist
in jedem Fall abhängig von der Zeit. Möge unsere Hochschule, die HTW Berlin, als Beispiel
dienen. Im Jahr 2017 hat die HTW mehrere Standorte in Berlin und auf jedem Standort existieren
mehrere Gebäude und Wege. Man kann daher ein Zuordnung. Es gibt eine Reihe von Geometrien,
die die Gebäude und Wege beschreiben. Man kann diese nun dem Objekt HTW zuweisen. Diese 
Zuweisung gilt aber für das Jahr 2017.

Vor einige Jahren entstand ein zusätzlicher Campus in Oberschöneweide, dafür wurden andere
Standorte abgegeben. Die Grundrisse der Gebäude haben sich nicht geändert. Die Zuordnung
der Geometrien zu den Geoobjekten haben sich aber geändert. 

Manche Geoobjekte änderten aber auch im Laufe der Zeit ihre Position 
(wie die Siegessäule in Berlin oder der Obelisk auf dem Place de la Concorde in Paris).
Viele Gebäude beherbergen über Jahre hinweg ganz unterschiedliche Objekte. Das aktuelle
Finanzministerium der Bundesregierung ist im Jahre 2017 im gleichen Gebäude in dem zuvor die
Treuhandanstalt war.

Anhand dieser Daten lassen sich historische Karten erzeugen. OHDM bietet eine WMS und WFS
Schnittstelle an, über die historische Karten angezeigt werden können, siehe auch http://ohdm.net

\subsection{Historische Daten}
OHDM erlaubt die Speicherung von Daten mit einem zeitlichen und örtlichen Bezug.
Aktuell arbeiten wir an der Integration von Sensordaten und von Daten, die einem Geoobjekt
zugeordnet werden können.

\subsubsection{Sensordaten}
Einige Arbeiten beschäftigen sich derzeit an Daten wie Temperatur, Feinstaubkonzentration, Luftfeuchtigkeit 
und Luftdruck. Sensordaten sind zeitliche und örtliche Größen. Messungen sind in der Regel 
punktuelle Messungen. Derzeit wird an Schnittstellen für den Import von Sensordaten in OHDM gearbeitet.
Die Visualisierung der Daten wird auch hier mittels WMS angeboten werden.

\subsubsection{Daten für historische Objekte}
Aktuell wird bei Geoobjekten vor allem an menschliche Infrastruktur gedacht (Häuser, Straßen etc.).
Administrative Objekte (Länder, Verwaltungsbezirke etc.) kommen hinzu.

Wir viele solcher Objekte lassen sich historische Dokumente (Bilder, Verweise auf Webseiten) finden.
Das können Bilder sein, aber auch Urkunden. Es können aber vor allem auch alte Karten sein.

Alte Karten sind in OHDM dann nutzbar, wenn sie gescannt vorliegen. Diese Rasterdaten können
Geoobjekten zugewiesen werden. Alte Karten können aber auch Basis für die Editoren sein, siehe 
Abschnitt \ref{editoren}.

Derzeit wird an einer Schnittstelle gearbeitet, die den Import solcher historischer Daten 
in OHDM erlaubt. Die Visualisierung wird als weiterer WMS-Layer in OHDM erfolgen.

\subsection{Importschnittstellen}
Derzeit wird an folgenden Importschnittstellen gearbeitet:

\begin{itemize}
\item
Import von GeoJSON
\item
Import von RDF-Files

\end{itemize}

\subsection{Anfrageschnittstellen}
Derzeit wird an folgenden Anfrageschnittstellen gearbeitet:

\begin{itemize}
\item
Linked Open Data Schnittstelle: Geoobjekte werden über URI referenzierbar.
\item
SPARQL-Endpoint
\item
GeoSPARQL-Endpoint
\item
Integration von CIDOC-CRM
\end{itemize}

